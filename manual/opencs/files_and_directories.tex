\section{Files and Directories}
\subsection{Introduction}
This section of the manual describes the directories and file types used by OpenCS. A file is a resource for storing data, identified by its
filename extension (e.g. .exe, .jpg, .txt), whereas a directory is a folder or file system structure in which these files are stored. You
are most likely already familiar with these concepts.

\subsection{Used terms} %TODO

\subsection{Basics}

\paragraph{Directories}
OpenMW and \OCS{} store their files in multiple directories. Firstly, there is the \textbf{user directory} that holds configuration
files and several other folders. The location of the user directory is hard coded for each supported operating system.

%TODO list paths.
In addition to the user directory, both \OMW{} and \OCS{} need a place to store the game’s actual data files: for example, the
textures, models, sounds and records of in-game objects. We support multiple paths to these files (termed \textbf{data paths}),
as specified in the configuration. Usually, one data path points to the directory where \MW{} is installed, however, you are
free to specify as many data paths as you would like. In addition, one particular data path, as described below, is used to store
newly created content files.

\paragraph{Content files}
\BS{} \MW{} engine uses two file types: ESM (master) and ESP (plugin). The distinction between the two is often confusing.
You would expect that the ESM (master) file is used to specify a single master which is modified by the ESP files (plugins), and indeed:
this is the basic idea. However, the original expansions are also ESM files, even though they can be described as very large plugins.
There were technical reasons behind this decision -- somewhat valid in the case of the original engine, but a more logical file system is
much preferable. \OMW{} achieves this through the creation of our own types of content file.

We support both ESM and ESP files, but, in order to make use of \OMW{}'s new features, one should consider using new file types designed
with our engine in mind: game files and addon files, collectively termed \textbf{content files}.

\subparagraph{OpenMW content files}
The distinction between game and addon files is similar to that between ESM and ESP, however their relationship to each other is
strictly defined -– the former are always master files, and the latter are always plugins. If you want to make a new game using the \OMW{}
engine (i.e. a ``total conversion''), you should create a game file. If you want to create an addon for an existing game file, simply
create an addon file. Nothing else matters: the only distinction you should consider is whether your project involves changing another game, 
or creating a new one. Simple as that.

Furthermore, our content files’ extensions are .omwaddon for addon files and .omwgame for game files.

%TODO describe what content files contains. and what not.
\subparagraph{\MW{} content files}
Using our content files is the recommended solution for projects that employ the \OMW{} engine. However, some players will wish to use 
the original \MW{} engine, despite its large flaws and lacking features\footnote{If this is wrong, we are a very successful project. Yay!}. 
In addition, since 2002 thousands of ESP/ESM files have been created, some with truly outstanding content. Because of this, \OCS{} is
committed to supporting ESP/ESM files. If you do decide to use ESP/ESM files rather than our own content files, you are most likely aiming
for original engine compatibility. This subject is covered in the very last section of the manual. 
%not finished TODO add the said section. Most likely when more features are present.

The actual creation of new files is described in the next chapter. Here we are going to focus only on the essential information needed
to create your first \OCS{} file. For now, let's jut remember that content files are stored in the user directory, in the \textbf{data}
subfolder (the particular data directory mentioned above).

\subparagraph{Dependencies}
Since addons aim to modify an existing game, it is logical that they also depend on the said game: otherwise they will not function.
For example, your modification changes the price of iron swords. But what if there are no iron swords in the game? That is right:
it is nonsense. Therefore, it is necessary to make your addon a dependency of other content files. These can be either game files
(e.g. an entirely new island), or other addon files (e.g. a house on the island). It is a good idea for addons to depend only on the
content files they modify, but this is up to the end user to determine.

Game files do not depend on any other content files, as they act as master files. A player can only use one game file at a time
(although this does not apply to the original and dirty ESP/ESM system).

%\subparagraph{Loading order} %TODO
\paragraph{Project files}
Project files contain data not used by the \OMW{} game engine but which are still needed by OpenCS. Good examples of this data type
are the record filters (described below). As a mod author, you probably do not need and/or want to distribute project files at all, 
as they are meant to be used only by you.

Since project files govern how content files are used in OpenCS, they are always used in conjunction with your specific project.
In fact, each time work commences on a content file that does not have a corresponding project file, a new project file will be created. 

The project file extension is ``.project''. The name of the project file is the whole name of the content file with appended extensions. 
For instance, a content file named swords.omwaddon is associated with the project file swords.omwaddon.project.

%TODO where are they stored.
Project files are stored inside the user directory, in the \textbf{projects} subfolder. This is both the location of newly created 
project files, and the place where \OCS{} looks for already existing files.

\paragraph{Resource files}
%textures, sounds, whatever
The vast majority of modern video games use what we shall term \textbf{resource files}: models, textures, icons, sounds and so on.
ESPs, ESMs and \OMW{} content files do not contain these files, merely instructions on how they are used. It follows that the \OMW{}
engine must be capable of supporting these resource files in order for them to function. Therefore this section covers ways to add 
resource files to your content file, and outlines which formats are supported. Later, you will learn how to make use of these files 
in your content.

\subparagraph{Audio}
OpenMW utilises {FFmpeg} for audio playback, so we support every audio type supported by this library. This is a huge list.
Below is only a small sample of supported file types.

\begin{description}
 \item mp3 ({MPEG}-1 {Part 3 Layer 3}) A popular audio file format and the \textit{de facto} standard for storing audio. Used by 
 the \MW{} game.
 \item ogg Open source, multimedia container file which uses the high quality vorbis audio codec. Recommended.
\end{description}

\subparagraph{Video}
As in the case of audio files, we use {FFmpeg} to decode video files. The list of supported files is long -– only the most 
significant will be covered.

\begin{description}
 \item bik Format used by the original \MW{} game.
 \item mp4 Multimedia container which use more advanced codecs ({MPEG-4 Parts 2,3,10}) with a better audio and video compression rate,
 but which require more {CPU} intensive decoding -- this probably makes it less suited for storing sounds in computer games, but
 good for videos.
 \item webm A new, shiny and open source video format with excellent compression. It needs quite a lot of processing power to be decoded,
 but since game logic is not running during cut scenes we can recommend it for use with \OMW.
 \item ogv An alternative, open source container using theora codec for video and vorbis for audio.
\end{description}

\subparagraph{Textures and images}
\MW{} uses {DDS} and {TGA} files for all kinds of two dimensional images and textures. In addition, the original engine supported BMP
files (although {BMP} is a terrible format for a video game). We also support an extended set of image files -- including {JPEG} and {PNG}.
JPEG and PNG files can be useful in some cases. For instance, a JPEG file is a valid option for a skybox texture and PNG can useful for masks.
However, keep in mind that a JPEG can grow large quickly and so are not the best option with a {DirectX} rendering backend. DDS files
are therefore recommended for textures. 
%\subparagraph{Meshes} %TODO once we will support something more than just nifs