\section{Filters}
\subsection{Introduction}
Filters are the key element of OpenCS use cases by allowing rapid and easy access to the seeked records presented in all tables. Therefore: in order to use this application fully effective you should make sure that all concepts and instructions written in the this section of the manual are perfectly clear to you.\\
Don't be afraid though, filters are fairly intuitive and easy to use.
\subsection{Used Terms}
\begin{description}
 \item[Filter] is generally speaking a tool able to ``Filter'' (that is: select some elements, while discarding others) according to the some criteria. In case of OpenCS: records are being filtred according to the criteria of user choice. Criteria are written down in language with simple syntax.
 \item[Criteria] describes condition under with any any record is being select by the filter.
 \item[Syntax] as you may noticed computers (in general) are rather strict, and expect only strictly formulated orders -- that is: written with correct syntax. Our syntax is simple and described in the {B}asics subsection.
 \item[Expression] is a criteria, only written with OpenCS filter syntax.
\end{description}
\subsection{Basics}
\subsection{Advanced Usage}