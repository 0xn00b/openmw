\section{Record filters}
\subsection{Introduction}
Filters are the key element of \OCS{} use cases by allowing rapid and easy access to the searched records presented in all tables.
Therefore: in order to use this application fully effective you should make sure that all concepts and instructions written in
the this section of the manual are perfectly clear to you.

Do not be afraid though, filters are fairly intuitive and easy to use.

\subsubsection{Used Terms}

\begin{description}
 \item[Filter] is generally speaking a tool able to ``Filter'' (that is: select some elements, while discarding others) according
 to the some criteria. In case of \OCS: records are being filtered according to the criteria of user choice. Criteria are written
 down in language with simple syntax.
 \item[Criteria] describes condition under with any any record is being select by the filter.
 \item[Syntax] as you may noticed computers (in general) are rather strict, and expect only strictly formulated orders -- that is:
 written with correct syntax.
 \item[Expression] is way we are actually performing filtering. Filter can be treated as ``functions'': accepts arguments, and evaluates
 either to the true or false for every column record at the time.
 \item[N-ary] is any expression that expects one or more expressions as arguments. It is useful for grouping two (or more) other expressions
 together in order to create filter that will check for criteria placed in two (again: or more) columns (logical \textit{or}, \textit{and}).
 \item[unary] is any expression that expects one other expression. The example is \textit{not} expression. In fact \textit{not} is the only useful
 unary expression in \OCS{} record filters.
 \item[nullary] is expression that does not accepts other expressions. It accepts arguments specified later.
\end{description}

\subsubsection{Basics}
In fact you do not need to learn everything about filters in order to use them. In fact all you need to know to achieve decent productivity
with \OCS{} is inside basics section.

\subsubsection{Interface}
Above each table there is a field that is used to enter filter: either predefined by the \OMW{} developers or made by you, the user.
You probably noticed it before. However there is also completely new element, although using familiar table layout. Go to the application
menu view, and click filters. You should see set of default filters, made by the \OMW{} team in the table with the following columns: filter,
description and modified.

\begin{description}
 \item[ID] contains the name of the filter.
 \item[Modified] just like in all other tables you have seen so far modified indicates if a filter was added, modified or removed.
 \item[Filter] column containing expression of the filter.
 \item[Description] contains the short description of the filter function. Do not expect any surprises there.
\end{description}

So let's learn how to actually use those to speed up your work.
\subsubsection{Using predefined filters}
Using those filters is quite easy and involves typing inside the filter field above the table. For instance, try to open referencables
table and type in the filters field the following: \mono{project::weapons}. As soon as you complete the text, table will magicly alter
and will show only the weapons. As you could noticed \mono{project::weapons} is nothing else than a~ID of one of the predefined filters. That is it:
in order to use the filter inside the table you simply type it is name inside the filter field.

To make life easier filter IDs follow simple convention. 

\begin{itemize}
 \item Filter ID filtering a specific record type contains usually the name of a specific group. For instance \mono{project::weapons} filter
 contains the word weapons (did you noticed?). Plural form is always used.
 \item When filtering specific subgroup the ID starts just like in the case of general filter. For instance \mono{project::weaponssilver} will
 filter only silver weapons (new mechanic introduced by the \BM{}, silver weapons deal double damage against werewolfs) and 
 \mono{project::weaponsmagical} will filter only magical weapons (able to hurt ghosts and other supernatural creatures).
 \item There are few exceptions from the above rule. For instance there is a \mono{project::added}, \mono{project::removed},
       \mono{project::modyfied}, \mono{project::base}.
 You would probably except something more like \mono{project::statusadded} but in this case typing this few extra characters would only
 help to break your keyboard faster.
\end{itemize}

We strongly recommend to take a look at the filters table right now to see what you can filter with that. And try using it! It is very simple.

\subsection{Advanced}
Back to the manual? Great.

If you want to create your own filter you have to know exactly what do you want to get in order to translate this into the expressions.
Finally, you will have to write this with correct syntax. As a result table will show only desired rows.

Advance subsection covers everything that you need to know in order to create any filter you may want to %TODO the filter part is actually wrong
\subsubsection{Namespaces}
Did you noticed that every default filter has \mono{project::} prefix? It is a \textit{namespace}, a~term borrowed from the \CPP{} language.
In case of \OCS{} namespace always means scope of the said object\footnote{You are not supposed to understand this at the moment.}.
But what does it mean in case of filters? Well, short explanation is actually simple.
\begin{description}
 \item[project::] namespace indicates that filter is used with the project, in multiple sessions. You can restart \OCS{} and filter
 is still there.
 \item[session::] namespace indicates that filter is not stored trough multiple sessions and once you will quit \OCS{} (close session)
 the filter will be gone. Forever! Until then it can be found inside the filters table.
\end{description}
In addition to this two scopes, there is a third one; called one-shot. One-shot filters are not stored (even during single session)
anywhere and as the name implies they are supposed to be created when needed only once. Good thing about the one-shot filters is that
you do not need to open filters table in order to create it. Instead you just type it directly inside the filter field, starting with
exclamation mark: ``!''.

Still, you may wonder how you are supposed to write expressions, what expressions you should use, and what syntax looks like. Let's start
with nullary expressions that will allow you to create a basic filter.

\subsubsection{Nullary expressions}
All nullary expressions are used in similar manner. First off: you have to write it is name (for instance: \mono{string}) and secondly:
condition that will be checked inside brackets (for instance \mono{string(something, something)}). If conditions of your expression will be meet
by a record (technical speaking: expression will evaluate to true) the record will show up in the table.

It is clear that you need to know what are you checking, that is: what column of the table contains information that you are interested
in and what should be inside specific cell inside this column to meet your requirements. In most cases first word inside brackets sets column
you want to see, while the second one sets desired value inside of the cell. To separate column argument from the value argument use comma.  

\paragraph{String -- string(``column'', ``value'')}
String in programmers language is often\footnote{Often, not always. There are different programming languages using slightly different terms.}
just a word for anything composed of characters. In case of \OCS{} this is in fact true for every value inside the column that is not composed
of the pure numbers. Even columns containing only ``true`` and ``false`` values can be targeted by the string expression\footnote{There is no
Boolean (``true'' or ``false'') value in the \OCS. You should use string for those.}. String evaluates to true, 
when record contains in the specified column exactly the same value as specified.

Since majority of the columns contain string values, string is among the most often used expressions. Examples:
\begin{itemize}
 \item \mono{string(``Record Type'', ``Weapon'')} -- will evaluate to true for all records containing \mono{Weapon} in the \mono{Record Type} column cell.
 This group contains every weapon (including arrows and bolts) found in the game.
 \item \mono{string(``Portable'', ``true'')} -- will evaluate to true for all records containing word true inside \mono{Portable} column cell.
 This group contains every portable light sources (lanterns, torches etc.).
\end{itemize}
This is probably enough to create around 90 string filters you will eventually need. However, this expression is even more powerful
-- it accepts regular expressions (also called regexps). Regular expressions is a way to create string criteria that will be matched 
by one than just one specific value in the column. For instance, you can display both left and right gauntlets with the following expression:
\mono{string("armor type", ".* gauntlet"))} because \mono{.*} in regexps means just: ``anything''. This filter says: please, show me ``any'' gauntlet.
There are left and right gauntlets in the morrowind so this will evaluate to true for both. Simple, isn't it?

Creating regexps can be a difficult and annoying -- especially when you need complex criteria. On the other hand, we are under impression
that in reality complex expressions are needed only in sporadic cases. In fact, the truth is: that most of the time only already mentioned
\mono{.*} is needed and therefore the following description of regexps can be skipped by vast majority of readers.

Before working with Regular Expressions, you should understand what actually are regular expressions. Essentially, the idea is simple:
when you are writing any word, you are using strictly defined letters -- that is: letters create a word. What you want to do with regular
expression is to use set of rules that will match to many words. It is not that difficult to see what it's needed to do so: first,
you will clearly need way to determinate what letters you want to match (word is composed by letters).

Before introducing other ways to choose between characters, I want explain anchors. Anchors allows you to decide where to ``look'' in the string.
You surely should know about \mono{\textasciicircum} anchor and \mono{\textdollar}. Putting \mono{\textasciicircum} will tell to \OCS{}
to look on the beginning of string, while \mono{\textdollar} is used to mark the end of it. For instance, pattern
\mono{\textasciicircum{}Pink.* elephant.\textdollar} will match any sentence beginning with the word \mono{Pink} and ending with
\mono{ elephant.}. Pink fat elephant. Pink cute elephant. It does not matter what is in between, because \mono{.*} is used.

You have already seen the power of the simple \mono{.*}. But what if you want to chose between only two (or more) letters? Well, this is when
\mono{[|]} comes in handy. If you write something like: \mono{\textasciicircum[a|k].*} you are simply telling \OCS{} to filter anything that
starts with either \mono{a} or \mono{k}. Using \mono{\textasciicircum[a|k|l].*} will work in the same manner, but it will also cover
strings starting with \mono{l}.

What if you want to match more than just one latter? Just use \mono{(|)}. it is pretty similar to the above one letter as you see, but it is
used to fit more than just one character. For instance: \mono{\textasciicircum(Pink|Green).* (elephant|crocodile).\textdollar} will be
true for all sentences starting with \mono{Pink} or \mono{Green} and ending with either \mono{elephant.} or \mono{crocodile.}.

Regular expressions are not the main topic of this manual. If you wish to learn more on this subject please, read the documentation on
Qt regular expressions syntax, or TRE regexp syntax (it is almost like in Qt). Above is just enough to use \OCS{} effectively to be sure.

\paragraph{Value -- value(``value'', (``open'', ``close''))}
While string expression covers vast group of columns containing string values, there are in fact columns with just numerical values like
``weight``. To filter those we need a value expression. This one works in similar manner to the string filter: first token name and criteria
inside brackets. Clearly, conditions should hold column to test in. However in this case wanted value is specified as a range.
As you would imagine the range can be specified as including a border value, or excluding. We are using two types of brackets for this:
\begin{itemize}
 \item To include value use [] brackets. For value equal 5, expression \mono{value(something, [5, 10])} will evaluate to true.
 \item To exclude value use () brackets. For value equal 5, expression \mono{value(something, (5, 10))} will evaluate to false.
 \item Mixing brackets is completely legal. For value equal 10, expression \mono{value(something, [5, 10)} will evaluate to true.
 The same expression will evaluate to false for value equal 10.
\end{itemize}

\paragraph{``true'' and ``false''}
Nullary \textit{true} and \textit{false} do not accept any arguments, and always evaluates to true (in case of \textit{true})
and false (in case of \textit{false})  no matter what. The main usage of this expressions is the give users ability to quickly
disable some part of the filter that makes heavy use of the logical expressions. 

\subsubsection{Logical expressions}
This subsection takes care of two remaining groups of expressions: binary and unary. The only unary expression present in the \OCS{} is logical
\textit{not}, while the remaining binary expressions are: \textit{or}, \textit{and}. This clearly makes them (from the user point of view)
belonging to the same group of logical expressions.

\paragraph{not -- not expression()}
Sometimes you may be in need of reversing the output of the expression. This is where \textit{not} comes in handy. Adding \textit{not} before
expression will revert it: if expression was returning true, it will return false; if it was returning false, it will return true. Brackets
are not needed: \textit{not} will revert only the first expression following it.

To show this on know example, let's consider the \mono{string("armor type", ".* gauntlet"))} filter. As we mentioned earlier this will return true
for every gauntlet found in game. In order to show everything, but gauntlets we simply do \mono{not string("armor type", ".* gauntlet"))}.
This is probably not the most useful filter on earth, but this is not a surprise: real value of \textit{not} expression shines when combined with
\textit{or}, \textit{and} filters.

\paragraph{or -- or(expression1(), expression2())}
\textit{Or} is a expression that will return true if one of the arguments evaluates to true. You can use two or more arguments, separated by the comma.

\textit{Or} expression is useful when showing two different group of records is needed. For instance the standard actor filter is using the following
\mono{or(string(``record type'', npc), string(``record type'', creature))} and will show both npcs and creatures.

\paragraph{and -- and(expression1(), expression2())}
\textit{And} is a expression that will return true if all arguments evaluates to true. As in the case of ``or'' you can use two or more arguments,
separated by a comma.
As we mentioned earlier in the \textit{not} filter, combining \textit{not} with \textit{and} can be very useful. For instance to show all armor types,
excluding gauntlets you can write the following: \mono{and (not string(``armor type'', ``.* gauntlet''), string(``Record Type'', ``Armor''))}.

\subsubsection{Creating and saving filter}
In order to create and save new filter, you should go to the filters table, right click and select option ``add record'' from the context menu.
A horizontal widget group at the bottom of the table should show up. From there you should select a namespace responsible for scope of
the filter (described earlier) and desired ID of the filter. After pressing OK button new entry will show up in the filters table. This filter
does nothing at the moment, since it still lacks expressions. In order to add your formula simply double click the filter cell of the new entry
and write it down there.
Done! You are free to use your filter.

\subsubsection{Replacing the default filters set}
OpenCS allows you to substitute default filters set provided by us, with your own filters. In order to do so you should create a new project,
add desired filters, remove undesired and save. Rename the file to the ``defaultfilters'' (do not forget to remove .omwaddon.project extension)
and place it inside your configuration directory.

The file acts as template for all new project files from now. If you wish to go back to the old default set, simply rename or remove the custom file.
