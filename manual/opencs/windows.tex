\section{Windows}
\subsection{Introduction}
This section describes the multiple windows interface of the \OCS{} editor. This design principle was chosen in order
to extend the flexibility of the editor, especially on the multiple screens setups and on environments providing advanced
windows management features, like for instance: multiple desktops found commonly on many open source desktop environments.
However, it is enough to have a single large screen to see the advantages of this concept.

OpenCS windows interface is easy to describe and understand. In fact we decided to minimize use of many windows concepts
applied commonly in various applications. For instance dialog windows are really hard to find in the \OCS. You are free to try,
though.

Because of this, and the fact that we expect that user is familiar with other applications using windows this section is mostly
focused on practical ways of organizing work with the \OCS.

\subsection{Basics}
After starting \OCS{} and choosing content files to use a editor window should show up. It probably does not look surprising:
there is a menubar at the top, and there is a~large empty area. That is it: a brand new \OCS{} window contains only menubar
and statusbar. In order to make it a little bit more useful you probably want to enable some panels\footnote{Also known as widgets.}.
You are free to do so, just try to explore the menubar.

You probably founded out the way to enable and disable some interesting tables, but those will be described later. For now, let's
just focus on the windows itself.

\paragraph{Creating new windows}
is easy! Just visit view menu, and use the ``New View'' item. Suddenly, out of the blue a new window will show up. As you would expect,
it is also blank, and you are free to add any of the \OCS{} panels.

\paragraph{Closing opened window}
is also easy! Simply close that window decoration button. We suspect that you knew that already, but better to be sure. 
Closing last \OCS{} window will also terminate application session.

\paragraph{Multi-everything}
is the main foundation of \OCS{} interface. You are free to create as many windows as you want to, free to populate it with 
any panels you may want to, and move everything as you wish to -- even if it makes no sense at all. If you just got crazy idea and
you are wonder if you are able to have one hundred \OCS{} windows showing panels of the same type, well most likely you are
able to do so.

The principle behind this design decision is easy to see for \BS{} made editor, but maybe not so clear for users who are
just about to begin their wonderful journey of modding.

\subsection{Advanced}
So why? Why this is created in such manner. The answer is frankly simple: because it is effective. When creating a mod, you often
have to work only with just one table. For instance you are just balancing weapons damage and other statistics. It makes sense
to have all the space for just that one table. More often, you are required to work with two and switch them from time to time.
All major graphical environments commonly present in operating systems comes with switcher feature, that is a key shortcut to change
active window. It is very effective and fast when you have only two windows, each holding only one table. Sometimes you have to work
with two at the time, and with one from time to time. Here, you can have one window holding two tables, and second holding just one.

OpenCS is designed to simply make sense and do not slowdown users. It is as simple as possible (but not simpler), and uses one
flexible approach in all cases.

There is no point in digging deeper in the windows of \OCS. Let's explore panels, starting with tables.

%We should write some tips and tricks here.