\section{Windows}
\subsection{Introduction}
This section describes the multiple windows interface of the OpenCS editor. This design principle was chosen in order to extend the flexibility of the editor, especially on the multiple screens setups and on environments providing advanced windows management features, like; for instance; multiple desktops found commonly on many open source desktop environments. However, it is enough to have a single large screen to see the advantages of this concept.\\
OpenCS windows interface is easy to describe and understand. In fact We decided to minimize use of many windows concepts applied commonly in various applications. For instance dialog windows are really hard to find in the OpenCS. You are free to try, though.\\
Because of this, and the fact that we expect that user is familiar with other applications using windows this section is mostly focused on practical ways of organizing work with the OpenCS.

\subsection{Basics}
After starting Open{CS} and choosing content files to use a editor window should show up. It probably does not look surprising: there is a menubar at the top, and there is a large empty area. That's it: a brand new Open{CS} window contains only menubar and statusbar. In order to make it a little bit more useful you probably want to enable some window widgets. You are free to do so, just try to explore the menubar. \\
You probably founded out the way to enable and disable some interesting tables, but those will be described later. For now, let's just focus on the windows itself.

\paragraph{Creating new windows}
is easy! Just visit view menu, and use the ``New View'' item. Suddenly, out of the blue a new window will show up. As you would expect, it is also blank, and you are free to add any of the Open{CS} widgets.